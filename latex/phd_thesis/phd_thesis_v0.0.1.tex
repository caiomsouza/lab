% Stanford University PhD thesis style -- modifications to the report style
% This is unofficial so you should always double check against the
% Registrar's office rules
% See http://library.stanford.edu/research/bibliography-management/latex-and-bibtex
% 
% Example of use below
% See the suthesis-2e.sty file for documentation
%
\documentclass{report}
\usepackage{suthesis-2e_ucm}
\dept{Statistics}

\usepackage{Sweave}
\begin{document}
\Sconcordance{concordance:phd_thesis_v0.0.1.tex:phd_thesis_v0.0.1.Rnw:%
1 12 1 1 0 116 1 1 2 44 1}

\title{Fuzzy linguistic models applied to the measurement and management of reputation in marketing in a Big Data environment.}
\author{Caio Fernandes Moreno}
\principaladviser{Ramon Alberto Carrasco Gonzalez}
\firstreader{First Reader Name}
\secondreader{Second Reader Name}

\beforepreface
\prefacesection{Preface}
This thesis tells you all you need to know about...
\prefacesection{Acknowledgments}
I would like to thank...
\afterpreface

\chapter{Introduction}


Test with formulas


The salary function:

 \begin{equation}
 f(x)  = rent \div 0.3 )
 \end{equation}
 
 
To estimate the salary: \(f(x) = \displaystyle \frac{rent}{0.3}\), we cannot say this is correct, but it can help people estimate how much money they need to earn based on a easy variable to get, the price to rent a house in a specific place. 
 


\chapter{Bayes' theorem}


Bayes' theorem


Sir Harold Jeffereys wrote that Bayes'theorem "is to the theory of probability what the Pythagorean theorem is to geometry" [1]

Bayes is a measure of belief. And it says that we can learn even from missing and inadequate data,
from approximations, and from ignorance. [2]

2.1 sssss

ssjsjssj


\chapter{Fuzzy linguistic models}

Fuzzy logic is a form of many-valued logic in which the truth values of variables may be any real number between 0 and 1. By contrast, in Boolean logic, the truth values of variables may only be the integer values 0 or 1. Fuzzy logic has been employed to handle the concept of partial truth, where the truth value may range between completely true and completely false.[1] Furthermore, when linguistic variables are used, these degrees may be managed by specific (membership) functions.[2]

The term fuzzy logic was introduced with the 1965 proposal of fuzzy set theory by Lotfi Zadeh.[3][4] Fuzzy logic had however been studied since the 1920s, as infinite-valued logic<U+2014>notably by <U+0141>ukasiewicz and Tarski.[5]

Fuzzy logic has been applied to many fields, from control theory to artificial intelligence.


\subsection*{Linguistic variables}

While variables in mathematics usually take numerical values, in fuzzy logic applications non-numeric values are often used to facilitate the expression of rules and facts.[6]

A linguistic variable such as age may accept values such as young and its antonym old. Because natural languages do not always contain enough value terms to express a fuzzy value scale, it is common practice to modify linguistic values with adjectives or adverbs. For example, we can use the hedges rather and somewhat to construct the additional values rather old or somewhat young.

Fuzzification operations can map mathematical input values into fuzzy membership functions. And the opposite de-fuzzifying operations can be used to map a fuzzy output membership functions into a "crisp" output value that can be then used for decision or control purposes.


\chapter{Place to tex Latex}

\texttt{ssss}

\begin{quote}
sssksksks dsfjsdf sfslkdfjsldk fjsldkj
\end{quote}


``ssssss ''

\begin{itemize}
  \item sss1
  \item sss 2
  \item sss3 ddd
\end{itemize}


\begin{verbatim}
ssss dsnfldsjflk sjkldf jsdlkfj slkd fjlks verbatim 
\end{verbatim}

\begin{enumerate}
  \item Big Data
  \item Data Science  
\end{enumerate}


\begin{description}
  \item[Big Data]
  \item[Data Science]
  \item[Machine Learning]
\end{description}




\chapter{Conclusions}

Write my conclusions.



\appendix
\chapter{A Long Proof}

Write the appendix.

\bibliographystyle{plain}
\bibliography{mybib}

  author        = {Arabacioglu, Burcin Cem},
  title         = {Using fuzzy inference system for architectural space analysis},
  journal       = {Applied Soft Computing},
  year          = {2010},
  volume        = {10},
  number        = {3},
  pages         = {926--937},
  publisher     = {Elsevier},
}

\chapter{Reference}

[1] Jeffreys, Harold (1973). Scientific Inference (3rd.), Cambridge University Press. p. 31. ISBN 978-0-521-18078-8.

APA : Jeffreys, H. (1973). Scientific inference. Cambridge University Press.

MLA : Jeffreys, Harold. Scientific inference. Cambridge University Press, 1973.

Chicago: Jeffreys, Harold. Scientific inference. Cambridge University Press, 1973.

Harvard: Jeffreys, H., 1973. Scientific inference. Cambridge University Press.

[2] MLA: McGrayne, Sharon Bertsch. The theory that would not die: how Bayes' rule cracked the enigma code, hunted down Russian submarines, XXXX emerged triumphant from two centuries of controversy. Yale University Press, 2011.

APA: McGrayne, S. B. (2011). The theory that would not die: how Bayes' rule cracked the enigma code, hunted down Russian submarines, XXXX emerged triumphant from two centuries of controversy. Yale University Press.


Fuzzy Logic (wikipedia)

[1] Arabacioglu, B. C. (2010). "Using fuzzy inference system for architectural space analysis". Applied Soft Computing. 10 (3): 926<U+2013>937. doi:10.1016/j.asoc.2009.10.011.

[2] Biacino, L.; Gerla, G. (2002). "Fuzzy logic, continuity and effectiveness". Archive for Mathematical Logic. 41 (7): 643<U+2013>667. doi:10.1007/s001530100128. ISSN 0933-5846.

[3] Cox, Earl (1994). The fuzzy systems handbook: a practitioner's guide to building, using, maintaining fuzzy systems. Boston: AP Professional. ISBN 0-12-194270-8.

[4] Gerla, Giangiacomo (2006). "Effectiveness and Multivalued Logics". Journal of Symbolic Logic. 71 (1): 137<U+2013>162. doi:10.2178/jsl/1140641166. ISSN 0022-4812.

[5] H<U+00E1>jek, Petr (1998). Metamathematics of fuzzy logic. Dordrecht: Kluwer. ISBN 0-7923-5238-6.

[6] H<U+00E1>jek, Petr (1995). "Fuzzy logic and arithmetical hierarchy". Fuzzy Sets and Systems. 3 (8): 359<U+2013>363. doi:10.1016/0165-0114(94)00299-M. ISSN 0165-0114.




\end{document}
